\documentclass[parskip=half]{scrartcl}
\usepackage[T1]{fontenc}
\usepackage[utf8]{inputenc}
\usepackage[ngerman]{babel}

\usepackage{libertine}% eine Schrift, die mir persönlich gut gefällt

\setkomafont{disposition}{\normalfont\bfseries}% Standardschrift für Überschriften

\usepackage{graphicx}% stellt \scalebox bereit
\usepackage{array,booktabs}% für Tabellen

\pagestyle{empty}% keine Seitenzahl

\begin{document}

\begin{center}
  \scalebox{5}{Vollmacht}
\end{center}

\vspace{1cm}

\begin{tabular}{@{}m{.5\linewidth}@{}>{\raggedleft}m{.5\linewidth}@{}}
  \begin{tabular}{@{}l@{}}
    \bfseries Vollmachtgeber \\
    David Stutz\\
    Hauptstraße 1\\
    11111 Berlin
  \end{tabular} &
  \begin{tabular}{@{}l@{}}
    \bfseries Vollmachtnehmer \\
    Max Mustermann\\
    Hauptstraße 1\\
    11111 Berlin
  \end{tabular}
\end{tabular}

\vspace{1cm}

\minisec{Vollmacht}

Hiermit bevollmächtige ich ...

Die Vollmacht gilt ab dem unten stehenden Datum bis die Vollmacht meinerseits
wieder eingezogen wird.

\vfill

\begin{tabular}{@{}m{.5\linewidth}@{}>{\raggedleft}m{.5\linewidth}@{}}
  \large\scshape Unterschrift: \\
  \midrule[3pt]
  & Ort und Datum, Name
\end{tabular}

\end{document}