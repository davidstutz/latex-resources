\documentclass[12pt,a4paper]{article}

\usepackage{amsmath}
\usepackage[font=footnotesize]{caption}
\usepackage[section]{algorithm}
\captionsetup[algorithm]{font=footnotesize}
\usepackage[numbered]{algo}

\begin{document}

	\begin{algorithm}[t]
		\begin{algo}{ERS}{\label{algo:related-work-ers}\qinput{undirected, weighted graph $G = (V, E)$}\qoutput{superpixel segmentation $S$}}
			initialize $M = \emptyset$\\
			\qforeach edge $(n,m) \in E$\\
				\qcom{Let $\hat{G}$ denote the graph $\hat{G} = (V, M \cup \{(n, m)\})$:}\\
				choose edge $(n,m) \in E$ yielding the largest gain in the energy $E(\hat{G})$\\
				\qif $\hat{G}$ has no cycles and $\hat{G}$ contains less or equal than $K$ connected components\\
					\qthen $M$ \qlet $M \cup \{(n,m)\}$\qfi\qrof\\
			\qcom{The superpixel segmentation is given by the connected components in $\hat{G}$:}\\
			derive superpixel segmentation $S$ from $\hat{G}$\\
			\qreturn $S$
		\end{algo}
		\caption{The greedy algorithm used to maximize the energy $E(\hat{G})$ to obtain Entropy Rate Superpixels \cite{LiuTuzelRamalingamChellappa}.}
		\label{fig:related-work-ers-algorithm}
	\end{algorithm}

	\begin{thebibliography}{1}
		\bibitem{LiuTuzelRamalingamChellappa}
		M. Y. Lui, O. Tuzel, S. Ramalingam, R. Chellappa.
		\emph{Entropy rate superpixel segmentation}.
		Converence on Computer Vision and Pattern Recognition, 2011.
	\end{thebibliography}

\end{document}